\documentclass[12pt]{article}
\usepackage{amsmath}
\usepackage{graphicx}
\usepackage{hyperref}
\usepackage{alltt}
\usepackage[utf8]{inputenc}
\usepackage[T1]{fontenc}
\usepackage[brazil]{babel}
\usepackage{graphicx}
\usepackage{fullpage}

\begin{document}

\thispagestyle{empty}

\newpage

\section*{Classe Abstrata {\sf Poligono}}

\begin{itemize}

\item Atributos: lados ({\sf vetor de elementos do tipo double})

\item Construtor único (que inicializa os lados do polígono) e Destrutor
  
\item Método {\sf getLado(int)} que retorna o tamanho do {\it i-ésimo} lado

\item Método {\sf getPerimetro()} que retorna o perímetro (soma dos lados)

\item Método {\sf getNroLados()} que retorna o número (quantidade) de lados

\item Método abstrato {\sf double getArea()} a ser implementado pelas subclasses

\item Sobrecarga do operador $<<$ 

\item Método  {\sf void imprime()} --  que imprime número de lados, perímetro e  área do polígono


(utilize o operador $<<$ implementado anteriormente)
  
\item Método estático  {\sf static bool comparaArea(Poligono* p1,  Poligono* p2)} --
  Comparação de 2 polígonos: retorna {\sf true}  se p1 é ``menor'' que p2 e {\sf
    false}, caso contrário. Deve-se levar  em consideração primeiramente a área.
  Se mesma área compara-se pelo perímetro.

\item Método estático  {\sf static bool comparaNroLados(Poligono* p1,  Poligono* p2)} --
  Comparação de 2 polígonos: retorna {\sf true}  se p1 é ``menor'' que p2 e {\sf
    false}, caso contrário. Deve-se levar  em consideração primeiramente o número de lados.
  Se mesmo número de lados compara-se pelo perímetro.
  
\end{itemize}

\par\noindent\rule{\textwidth}{0.4pt}

\begin{quote}
\begin{scriptsize}
\begin{verbatim}
#ifndef POLIGONO_H
#define POLIGONO_H

#include <iostream>
#include <vector>
using namespace std;

class Poligono {
public:
    Poligono(vector<double>&);
    virtual ~Poligono();
    double getLado(int) const;
    double getPerimetro() const;
    virtual double getArea() const = 0;
    virtual void imprime() const ;
    friend ostream& operator<<(ostream&, const Poligono&);

    static bool comparaArea(const Poligono* p1, const Poligono* p2);
    static bool comparaNroLados(const Poligono* p1, const Poligono* p2);
private:
    vector<double>& lados;
};

#endif /* POLIGONO_H */
\end{verbatim}
\end{scriptsize}
\end{quote}

\par\noindent\rule{\textwidth}{0.4pt}

\newpage

\section*{Classe {\sf Triângulo} (subclasse de {\sf Poligono}}

\begin{itemize}

\item Método {\sf double getArea()} -- que retorna a área do triângulo.

$A = \displaystyle\sqrt{S * (S - lado_1) * (S - lado_2) * (S - lado_3)}$, onde
$S=\displaystyle\frac{Perimetro}{2}$\\

\end{itemize}

\par\noindent\rule{\textwidth}{0.4pt}

\begin{quote}
\begin{scriptsize}
\begin{verbatim}
#ifndef TRIANGULO_H
#define TRIANGULO_H

#include "Poligono.h"

class Triangulo : public Poligono {
public:
    Triangulo(vector<double>&);
    virtual ~Triangulo();
    virtual double getArea() const;
private:
};

#endif /* TRIANGULO_H */
\end{verbatim}
\end{scriptsize}
\end{quote}

\par\noindent\rule{\textwidth}{0.4pt}

\section*{Classe {\sf Retangulo} (subclasse de {\sf Poligono})}

\begin{itemize}

\item Método {\sf double getArea()} -- que retorna a área do triângulo.

$A = \displaystyle lado_1 * lado_2$ 
\end{itemize}

\par\noindent\rule{\textwidth}{0.4pt}

\begin{quote}
\begin{scriptsize}
\begin{verbatim}
#ifndef RETANGULO_H
#define RETANGULO_H

#include "Poligono.h"

class Retangulo :public Poligono {
public:
    Retangulo(vector<double>&);
    virtual ~Retangulo(); 
    virtual double getArea() const;
private:
};

#endif /* RETANGULO_H */
\end{verbatim}
\end{scriptsize}
\end{quote}

\par\noindent\rule{\textwidth}{0.4pt}

\section*{Programa Principal}

Crie o arquivo {\sf main.cpp} com o seguinte conteúdo:

\par\noindent\rule{\textwidth}{0.4pt}

\begin{quote}
\begin{scriptsize}
\begin{verbatim}
#include <iostream>
#include <algorithm> // std::sort
#include <vector>    // std::vector
#include "Poligono.h"
#include "Triangulo.h"
#include "Retangulo.h"

int main() {

    vector<Poligono *> poligonos;

    vector<double> v1{3, 4, 5};
    poligonos.push_back(new Triangulo(v1));

    vector<double> v2{3, 4, 3, 4};
    poligonos.push_back(new Retangulo(v2));

    vector<double> v3{3, 3, 3};
    poligonos.push_back(new Triangulo(v3));

    vector<double> v4{2, 3, 2, 3};
    poligonos.push_back(new Retangulo(v4));

    cout << "poligonos:" << endl;

    for (unsigned long int i = 0; i < poligonos.size(); i++) {
        cout << *poligonos[i] << endl;
    }

    cout << endl << "poligonos (ordenado pela Area):" << endl;

    sort(poligonos.begin(), poligonos.end(), Poligono::comparaArea);

    for (unsigned long int i = 0; i < poligonos.size(); i++) {
        poligonos[i]->imprime();
    }

    cout << endl << "poligonos (ordenado pelo Nro Lados):" << endl;

    sort(poligonos.begin(), poligonos.end(), Poligono::comparaNroLados);

    for (unsigned long int i = 0; i < poligonos.size(); i++) {
        poligonos[i]->imprime();
    }

    return 0;
}
\end{verbatim}
\end{scriptsize}
\end{quote}

\par\noindent\rule{\textwidth}{0.4pt}

\end{document}


