\documentclass[11pt]{article}

\usepackage[brazil]{babel} 
\usepackage[latin1]{inputenc} 
\usepackage{alltt}
\usepackage{multicol} 
\usepackage{graphicx}
\usepackage{listings}
\usepackage{amsmath, amssymb}

\setlength{\topmargin}{-3cm} 
\setlength{\textheight}{27.5cm}
\setlength{\oddsidemargin}{-2cm} 
\setlength{\evensidemargin}{-2cm}
\setlength{\textwidth}{7.5in}

\newcommand{\comb}[2]
%%  to be used in math mode 
{\left( \begin{array}{c} #1 \\ #2 \end{array} \right) }
 
\def\ni{\noindent}

\def\idc{\makebox[1.5cm]{}}

\def\espaco{\makebox[5mm]{}}
 
\def\ua{\uparrow}
 
\def\pule{\vspace{0.2cm}} \def\pulao{\vspace{0.5cm}}
 
\newcommand{\primeira}[1] {$#1^{\mbox{\scriptsize\b{a}}}$}
 
\newcommand{\primeiro}[1] {#1$^{\mbox{\scriptsize\b{o}}}$}
 
\begin{document}
\lstset{language=Java}
\lstset{frameround=fttt}
\thispagestyle{empty}

\begin{enumerate}

\item Um n�mero  complexo $z$ pode ser  escrito na forma: $z =  a + bi$,
  onde  $a,  b \in  \mathbb{R}$,  $z  \in \mathbb{C}$  e  $i$  denota a  unidade
  imagin�ria.

  Defina a  classe {\sf  Complexo} que representa  n�meros complexos.  Inclua na
  classe um construtor �nico capaz de setar os atributos.

\begin{itemize}

\item Implemente (Sobrecarregue) o operador $+$ na classe {\sf Complexo}.

  Sejam $z_1 = a + bi$ e $z_2 = c + di$. A soma $z_1 + z_2 = (a + c)
  + (b + d)i$.
  
  Exemplo: $z_1 = 1 + 2i$ e $z_2 = 3 + 4i$. A soma $z_1 + z_2 = 4 + 6i$.

\item Implemente (Sobrecarregue) o operador $-$ (un�rio) na classe {\sf Complexo}.

  Seja $z = a + bi.$ A aplica��o do operador $-$ (un�rio) em $z$ resulta $\Rightarrow -z = -a + (-b)i$
  
  Exemplo: $z = 1 + 2i \Rightarrow -z = -1 - 2i$.

\item Implemente (Sobrecarregue) o operador $-$ na classe {\sf Complexo}.

  Sejam $z_1 = a + bi$ e $z_2 = c + di$. A subtra��o $z_1 - z_2 = (a - c)
  + (b - d)i$.
  
  Exemplo: $z_1 = 1 + 2i$ e $z_2 = 3 + 4i$. A subtra��o $z_1 - z_2 = -2 - 2i$.

\item Implemente (Sobrecarregue) o operador $*$ na classe {\sf Complexo}.

  Sejam $z_1 =  a + bi$ e  $z_2 = c +  di$. A multiplica��o $z_1 * z_2 = ac  - bd +
  (ad + bc)i$.

  Exemplo: $z_1 = 1 + 2i$ e $z_2 = 3 + 4i$. A multiplica��o $z_1 * z_2 = -5 + 10i$.

\item Implemente (Sobrecarregue) o operador $/$ na classe {\sf Complexo}.

  Sejam $z_1 =  a + bi$ e  $z_2 = c +  di$. A divis�o $z_1 / z_2 = \frac{ac  + bd}{c^2 + d^2} + \frac{bc - ad}{c^2 + d^2}i$.

  Exemplo: $z_1 = 1 + 2i$ e $z_2 = 3 + 4i$. A divis�o $z_1 / z_2 = \frac{11}{25} + \frac{2}{25}i$.
  
\item Implemente  (Sobrecarregue) o operador  $<<$ na classe {\sf  Complexo}.  Esse
  operador deve possibilitar a  impress�o (no formato $a + bi$)  das informa��es de 
  um n�mero complexo.
\end{itemize}

\end{enumerate}

\end{document}
